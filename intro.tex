\documentclass[main.tex]{subfiles}
\begin{document}

In high energy particle collisions, such as performed at the Large Hadron Collider at CERN, protons or heavy ions are collided at high energies in order to study the structure of their underlying matter. \\
In this way, it has been found that the colliding hadrons, such as protons and neutrons, consist of smaller particles, called partons.
At the energies reached at the Large Hadron Collider, the colliding particles can surmount each others repelling potential such that hadron collisions are effectively the composition of several parton-parton sub-collisions. The partons, carrying colour charge, are subject to the strong interaction, which is responsible for most of the particle production in particle collisions. The unique properties of the strong interaction cause the elementary interaction between two partons to result in the production of several new hadrons. In experiments, such as the ALICE experiment at CERN, the kinematic properties of the produced particles are measured and analysed in order to draw conclusions on the mechanisms responsible for their production. \\
In QCD, the quantum field theory describing strong interactions, interactions at large momentum scales can be calculated using perturbation theory. The processes responsible for the formation of new hadrons, i.e. the hadronisation, however, typically include small momentum transfers. In these cases phenomenological models have to be employed that can explain the observed phenomena. Since these models can not be derived from first principles, a lot of uncertainties still remain in our understanding of the hadronisation processes. 

In this work the final state properties of charged-particles, in particular their transverse momentum, and the total number of produced particles, measured with \mbox{ALICE}, are analysed in order to study the underlying mechanisms of hadronisation. \\
In the first part of this thesis, the current models of hadronisation that are implemented in so-called Monte Carlo event generators, are compared to recent proton-proton as well as proton-lead collisions. \\
In the second part of this thesis the particles are analysed in different azimuthal regions, defined with respect to the highest transverse-momentum particle produced in the collision, the leading particle. The region transverse to the leading particle can then be identified with the so-called underlying event, a convolution of several parton-parton sub-collision producing mostly soft particles that underlie the particle jets, produced in the hardest sub-collision that also produced the leading particle. \\
In chapter 2, an introduction to the strong interaction and the fundamentals of particle collisions as well as their description in Monte Carlo event generators is given. In chapter 3, the ALICE apparatus is shortly discussed with focus on the Time Projection Chamber, used for the measurements analysed in this thesis. Subsequently, an introduction to the analysed observables, in particular the mean transverse momentum, and the underlying event, is given in chapter 4. In chapter 5 some information on the analysed datasets are given. In Chapter 6 and 7 the results of the Monte Carlo comparison as well as the analysis in the underlying event, as described above, are discussed. Finally a summary is given in chapter 8.

\end{document}