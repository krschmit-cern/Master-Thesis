\documentclass[main.tex]{subfiles}
\begin{document}

The underlying event constitutes of all particles and interactions that do not derive directly from the hardest parton-parton collision, also called primary collision. This includes additional parton-parton collisions and their resulting interactions as well as initial and final state radiation. The hadronisation of the MPI systems and their interactions can not be derived from first principle and is instead described using phenomenological models. \\
In chapter \ref{Rivet} the comparison of two distinct MCEG \pythia{8.24} respectively \angantyr{} and \epos{} has shown deviations in the description of particle collisions from measurement. The analysis of transverse momentum and multiplicity distributions of the underlying event might offer additional insight in the processes responsible for particle production. 
In the following, it is therefore attempted to analyse the transverse momentum and charged-particle multiplicity in three different angular regions as described in chapter \ref{obs:underlying_event}, and use the transverse region as a proxy of the underlying event.

\section{Leading Particle}

In practice, it is not known a priori which particles result from the primary collision and which compose the underlying event. As explained in chapter \ref{obs:underlying_event} the direction of the primary jet, resulting from the primary collision, is approximated by the particle with the highest transverse momentum, the leading particle. The event is then divided into three parts with equal azimuthal coverage, see left part of figure \ref{obs:regions_wheel}. It is assumed that the primary jet is confined within the towards and away regions, so that the transverse region only contains underlying-event particles. Note, however, that the towards and away regions may still contain underlying event particles. \\
This assumption is only valid above a certain value of the leading particle transverse momentum. Above this threshold the (mean) transverse multiplicity is independent of the leading particle and reaches a plateau (underlying event plateau). The underlying event plateau for this analysis can be seen in the left hand side of figure \ref{ue:leading_track}. As can be seen from the figure, the leading track transverse momentum cut is set to be at 4 GeV/\textit{c}. The right hand side of figure \ref{ue:leading_track} depicts the corresponding probability distribution of the leading track transverse momentum. The figure shows that roughly 99\% of the previously 300 million events will be lost through the cut. 

\begin{figure*}[t!]
    \centering
	\begin{subfigure}{0.495\textwidth}
		\centering
		\includegraphics[width=\textwidth]{Analysis/img/ue/leading_particle/PlateauMB.pdf}
	\end{subfigure}
	\hfill
	\begin{subfigure}{0.495\textwidth}
		\centering
		\includegraphics[width=\textwidth]{Analysis/img/ue/leading_particle/LeadPTMB.pdf}
	\end{subfigure}
    \caption{Mean multiplicity in transverse region as function of leading track transverse momentum, reaching a plateau (underlying event plateau) at 4 GeV/\textit{c} (left) and probability distribution of leading track transverse momentum (right). The leading transverse momentum cut is indicated respectively.}
    \label{ue:leading_track}
    \vspace*{\floatsep}
    \centering
	\begin{subfigure}{0.495\textwidth}
		\centering
		\includegraphics[width=\textwidth]{Analysis/img/ue/leading_particle/DeltaPhi.pdf}
	\end{subfigure}
	\hfill
	\begin{subfigure}{0.495\textwidth}
		\centering
		\includegraphics[width=\textwidth]{Analysis/img/ue/leading_particle/DeltaPhiMB.pdf}
	\end{subfigure}
    \caption{Probability distribution of angle between any particle and the corresponding leading particle for events with applied leading particle transverse momentum cut (left) and without (right)}
    \label{ue:delta_phi}
\end{figure*}

The leading transverse momentum cut also introduces a bias in the towards, and due to the back-to-back jets possibly in the away region to the data. Figure \ref{ue:delta_phi} shows the probability distribution of the angle between an arbitrary particle and the leading particle ($\Delta\varphi = \varphi_\text{lead} - \varphi_\text{particle})$ with (left) and without (right) leading transverse momentum cut. The wavelike structure superimposed in the distribution with leading \textit{p}\textsubscript{T} cut is caused by ??????, see appendix \ref{app:waves}. \\
The towards side is always the region containing the highest transverse momentum particle, it is therefore expected that the number of produced particles in this region is enhanced, compared to the other two regions. This is confirmed by the right side of figure \ref{ue:delta_phi}. It can be seen that with introducing the leading transverse momentum cut the probability of finding a particle in the towards region is enhanced by roughly 30\% compared to the distribution without cut. Since momentum must be conserved in each event and each interaction respectively it is also expected that the particle production in the away region is also slightly enhanced, compared to the transverse region, as is confirmed by figure \ref{ue:delta_phi} as well.\\
In the following all results will be shown with as well as without the leading transverse momentum cut applied, so that the effects of the cut are visible. Note, however, that in the case where no leading $p_\text{T}$ cut is applied, the transverse region is contaminated with wide angle radiation of the leading jet, and can not be identified as pure underlying event.

\section{Multiplicity}

    
\begin{figure}
    \centering
	\begin{subfigure}{0.495\textwidth}
		\centering
		\includegraphics[width=0.95\textwidth]{Analysis/img/ue/event_vars/RT.pdf}
	\end{subfigure}
	\hfill
	\begin{subfigure}{0.495\textwidth}
		\centering
		\includegraphics[width=0.95\textwidth]{Analysis/img/ue/event_vars/RTMB.pdf}
	\end{subfigure}
	\begin{subfigure}{0.495\textwidth}
		\centering
		\includegraphics[width=0.95\textwidth]{Analysis/img/ue/event_vars/multRatio.pdf}
	\end{subfigure}
	\hfill
	\begin{subfigure}{0.495\textwidth}
		\centering
		\includegraphics[width=0.95\textwidth]{Analysis/img/ue/event_vars/multRatioMB.pdf}
	\end{subfigure}
    \caption{Multiplicity distributions (top row) in the corresponding azimuthal regions for events with leading particle $p_\text{T}$ cut applied (left) and without (right), as well as the corresponding ratios to \textit{all regions} and \textit{Minimum Bias} respectively (bottom row). All distributions have been scaled to their corresponding azimuthal range. The systematic uncertainties are shown as boxes, the statistical uncertainties as bars.}
    \label{ue:mult_ratio}
    
\end{figure}

The upper panels of figure \ref{ue:mult_ratio} shows the corresponding charged-particle multiplicity distributions for each of the three regions as well as all of them combined (all regions) for events with as well as without leading particle \pt{} cut. All distributions have been scaled to their corresponding azimuthal coverage, the unscaled distributions can be found in appendix \ref{app:unscaled}. The lower panels of figure \ref{ue:mult_ratio} show the corresponding ratios to \textit{all regions} and \textit{Minimum Bias} respectively. In the case where no leading \pt{} cut has been applied, the shape of the regions are very similar, as can be seen from the ratios. The corresponding mean and variance as well as the position and value of the maxima are shown in figure \ref{ue:mult_moments}. It can be seen that higher multiplicities in the towards and away side are slightly enhanced in comparison to the transverse side. \\
In contrast, the distributions obtained from events where the leading \pt{} cut has been applied show clearly distinct maxima and mean values, as can be seen from figure \ref{ue:mult_ratio}. The towards region is strongly enhanced in comparison to the away and the transverse region. The variance and therefore the width of the distributions are, however quite similar in the three regions. This indicates that the underlying event fills up the whole angular range of the event, and the hard collision contributes additional particles in the towards and the away region, as assumed above. Additionally it can be concluded that the leading jet contributes a large fraction to the total multiplicity, otherwise the hard collision would be negligible in comparison to the total multiplicity. \\
The $N_\text{ch}$ distributions of events selected with the leading particle \pt{} cut are generally shifted to higher multiplicities in comparison to the corresponding distributions without cut. This is a consequence of the impact parameter dependence as explained above. The leading \pt{} cut biases the event to smaller impact parameters, i.e. bigger overlap of the colliding hadrons. This results in higher probabilities for hard interactions as well as for more MPI in one event, both leading to enhanced particle production. 

\begin{figure}     
    \centering
	\begin{subfigure}{0.495\textwidth}
		\centering
		\includegraphics[width=0.95\textwidth]{Analysis/img/ue/event_vars/multmean.pdf}
	\end{subfigure}
	\hfill
	\begin{subfigure}{0.495\textwidth}
		\centering
		\includegraphics[width=0.95\textwidth]{Analysis/img/ue/event_vars/multmax.pdf}
	\end{subfigure}
    \caption{Mean  and  Variance  (left)  as  well  as  position  and  value  of  maximum(right) of charged-particle multiplicity measured in the towards, away and transverse region as well as \textit{all regions} combined. }
    \label{ue:mult_moments}
\end{figure}

In order to approximate the contribution of the leading jet to the total event multiplicity, in the following the fraction of particles measured in one of the  regions is considered as a function of the total event multiplicity. For this purpose the average fraction Z of any observable measured in a specific region to the event total is defined as follows:

\begin{equation}
    Z(x; y_\text{total}) = \frac{[\text{mean} (x_\text{region})]_{y_\text{total}}}{x_\text{total}}  \label{eq:ZED}
\end{equation}

The parameter $y_\text{total}$ denotes a condition on the events considered in the calculation, e.g. that the total number of charged particles in the event has to be equal to $x_\text{total}$.
The fraction of $N$ measured in e.g. the towards region to the total measured multiplicity as function of the total multiplicity is then given by:

\begin{equation}
    Z(N_\text{ch}; x_\text{total}) = \frac{[\text{mean} (N_\text{region})]_{ x_\text{total}}}{x_\text{total}}  
\end{equation}

\begin{figure}
    \centering
	\begin{subfigure}{0.495\textwidth}
		\centering
		\includegraphics[width=0.95\textwidth]{Analysis/img/ue/correlations/multComp_all.pdf}
	\end{subfigure}
    \caption{Fraction of the measured raw multiplicity $N_\text{acc}$ in each region as function of the total $N_\text{acc}$, with and without the leading \pt{} cut applied respectively.}
    \label{ue:mult_percent}
\end{figure}

The corresponding fractions Z of the measured raw multiplicity $N_\text{acc}$, i.e. not unfolded with the procedure as described in \ref{dataset:unfolding}, for the four azimuthal regions with and without leading \pt{} cut respectively can be seen in figure \ref{ue:mult_percent}. Both of the event samples show a similar trend. At small total $N_\text{acc}$, where there is very little or no underlying event, the towards region strongly predominates. Towards higher $N_\text{acc}$, where the MPI start contributing to the multiplicity, the fractions get closer to each other until they are roughly equal within their statistical uncertainties at a value of 33\%. At this point the contribution of the hard collision becomes negligible in comparison to the total multiplicity and all of the regions contribute roughly evenly. In the event sample where the leading \pt{} cut is applied, the towards and away region are slightly enhanced in comparison to the distributions without the cut applied, especially at low $N_\text{acc}$. \\
The fraction of particles produced by the hard collision can be approximated from the distributions with leading \pt{} cut applied. At intermediate $N_\text{acc}$, where the transverse region contains approximately 30\% of all particles, the hard collision contributes roughly 10\% to the total multiplicity, assuming that the underlying event is evenly distributed in all of the regions. 

\section{Transverse momentum spectra}


\begin{figure*}[t!]
    \centering
	\begin{subfigure}{0.495\textwidth}
		\centering
		\includegraphics[width=0.95\textwidth]{Analysis/img/ue/pT_spectra/pT.pdf}
	\end{subfigure}
	\hfill
	\begin{subfigure}{0.495\textwidth}
		\centering
		\includegraphics[width=0.95\textwidth]{Analysis/img/ue/pT_spectra/pTMB.pdf}
	\end{subfigure}
    % \vspace*{\floatsep}
    \centering
	\begin{subfigure}{0.495\textwidth}
		\centering
		\includegraphics[width=0.95\textwidth]{Analysis/img/ue/pT_spectra/pTRatio.pdf}
	\end{subfigure}
	\hfill
	\begin{subfigure}{0.495\textwidth}
		\centering
		\includegraphics[width=0.95\textwidth]{Analysis/img/ue/pT_spectra/pTRatioMB.pdf}
	\end{subfigure}
    \caption{Transverse momentum distributions (top row) in the corresponding azimuthal regions for events with (left) and without (right) leading $p_\text{T}$ cut applied, as well as the corresponding ratios to \textit{all regions} and \textit{Minimum Bias} respectively (bottom row). All distributions have been scaled to their corresponding azimuthal range. The systematic uncertainties are shown as boxes, the statistical uncertainties as bars.}
    \label{ue:pt}
\end{figure*}

Figure \ref{ue:pt} shows the \pt{} distributions of the different azimuthal regions for event samples with as well as without leading \pt{} cut applied in the upper panels, and the corresponding ratios to \textit{all regions} and \textit{Min. Bias} in the lower panels. The distributions have been scaled to the corresponding azimuthal ranges, the unscaled distributions can be found in appendix \ref{app:unscaled}. In the distributions without the leading \pt{} cut applied, the towards side is clearly enhanced at higher transverse momenta in comparison to the other regions as expected. The away and the transverse side, however overlap up to $p_\text{T} \approx 1$, for higher \pt{} the distributions separate and the away side is enhanced at high \pt{} in comparison to the transverse side. Since the multiplicity distributions, see figure \ref{ue:mult_ratio}, have roughly the same shape with the towards side being slightly enhanced it can be discerned that the enhancement of the towards and away side at high \pt{} is due to the contributions of the hard collision. In the distributions for which the leading \pt{} cut has been applied this effect is clearly visible. The towards, away and transverse side overlap for small \pt{}, meaning that the soft particles are distributed evenly throughout the azimuthal range. This supports the assumption that the leading jet (pair) superimposes on an evenly distributed underlying event. The steps occurring at $p_\text{T} = 4$ in the towards as well as the \textit{all regions} distributions is due to the leading particle \pt{} cut that enforces an enhancement of the higher \pt{} values. From the enhancement of the away region however, it can be discerned that the primary collision contributes mostly hard particles to the final state of the event.

\subsection{Mean \pt{} as function of \mult{}}

As mentioned above, the transverse momentum and the multiplicity of an event are closely tied together by the hadronisation processes. In the following the correlation between the two will be examined by studying the mean as well as the variance of the transverse momentum distributions as a function of the charged particle multiplicity. 

\begin{figure*}[t!]
    \centering
	\begin{subfigure}{0.495\textwidth}
		\centering
		\includegraphics[width=0.95\textwidth]{Analysis/img/ue/pT_spectra/meanRT.pdf}
	\end{subfigure}
	\hfill
	\begin{subfigure}{0.495\textwidth}
		\centering
		\includegraphics[width=0.95\textwidth]{Analysis/img/ue/pT_spectra/meanRTMB.pdf}
	\end{subfigure}
    % \vspace*{\floatsep}
        \centering
	\begin{subfigure}{0.495\textwidth}
		\centering
		\includegraphics[width=0.95\textwidth]{Analysis/img/ue/pT_spectra/meanRatio.pdf}
	\end{subfigure}
	\hfill
	\begin{subfigure}{0.495\textwidth}
		\centering
		\includegraphics[width=0.95\textwidth]{Analysis/img/ue/pT_spectra/meanRatioMB.pdf}
	\end{subfigure}
    \caption{mean $p_\text{T}$ as function of $N_\text{acc}$ (top  row) in the corresponding azimuthal regions for events with (left) and without (right) leading \pt{} cut applied, as well as the corresponding ratios to \textit{all regions} and \textit{Minimum Bias} respectively (bottom row). All distributions have been scaled to their corresponding azimuthal range.  The systematic uncertainties are shown as boxes, the statistical uncertainties as bars.}
    \label{ue:mean}
\end{figure*}

Figure \ref{ue:mean} shows the mean transverse momentum as a function of multiplicity for the three regions as well as \textit{all regions} combined for event samples with as well as without the leading particle \pt{} cut applied in the upper panels. The lower panels show the corresponding ratios to \textit{all regions} and \textit{Min. Bias} respectively. The distributions have been scaled to their azimuthal coverage, the unscaled distributions can be found in appendix \ref{app:unscaled}. For the event sample without the leading \pt{} cut applied, the mean \pt{} distributions of the away and the transverse region are overlapping almost completely, as can also be seen in the ratio. This is probably due to the wide angle radiation of the primary collision contaminating the transverse region, since the distributions where the leading particle \pt{} is applied show a different behaviour. Although the shape of the distributions is quite similar, the mean \pt{} of the away region is enhanced by about 15\% compared to the distribution in the transverse region. This supports that the underlying mechanisms of the hadronisation of the MPI and the primary collision are the same, but they include different momentum scales. \\
In the towards side of the event sample without the leading \pt{} cut, mean \pt{} is enhanced by roughly 20\% to 40\% in comparison to the MB data at low multiplicities. Towards higher multiplicities, the mean \pt{} in the towards region gets closer to the values in the away and the transverse side, until they overlap at $N_\text{ch} \approx 17$, i.e. $N_\text{ch}/(2\pi\Delta\varphi) \approx 50$. At this point the enhancement through the primary collision becomes negligible in comparison to the total number of particles. It is expected that this trend continues towards higher multiplicities, it is however not possible to draw any conclusion regarding this from these measurements due to the large systematic uncertainties.
The distribution in the towards region where the leading \pt{} cut has been applied is strongly biased to high values of mean \pt{} up to multiplicities $N_\text{ch} \approx 12$, i.e. $N_\text{ch}/(2\pi\Delta\varphi) \approx 35$. For higher \mult{}, the distribution approaches the mean \pt{} values in the away and the transverse side. It is expected that the distributions continue to approach each other until they eventually overlap, similar to the case where the leading \pt{} cut has not been applied. These measurements, however, have insufficient statistics to confirm this hypothesis.


\subsection{Variance \pt{} as function of \mult{}}

\begin{figure}
        \centering
	\begin{subfigure}{0.495\textwidth}
		\centering
		\includegraphics[width=0.95\textwidth]{Analysis/img/ue/pT_spectra/varRT.pdf}
	\end{subfigure}
	\hfill
	\begin{subfigure}{0.495\textwidth}
		\centering
		\includegraphics[width=0.95\textwidth]{Analysis/img/ue/pT_spectra/varRTMB.pdf}
	\end{subfigure}
	
	 \centering
	\begin{subfigure}{0.495\textwidth}
		\centering
		\includegraphics[width=0.95\textwidth]{Analysis/img/ue/pT_spectra/varRatio.pdf}
	\end{subfigure}
	\hfill
	\begin{subfigure}{0.495\textwidth}
		\centering
		\includegraphics[width=0.95\textwidth]{Analysis/img/ue/pT_spectra/varRatioMB.pdf}
	\end{subfigure}
    \caption{variance of $p_\text{T}$ as function of $N_\text{acc}$ (top  row) in the corresponding azimuthal regions for events with (left) and without (right) leading \pt{} cut applied, as well as the corresponding ratios to \textit{all regions} and \textit{Minimum Bias} respectively (bottom row). All distributions have been scaled to their corresponding azimuthal range.  The systematic uncertainties are shown as boxes, the statistical uncertainties as bars.}
    \label{ue:var}
\end{figure}

Additionally to the mean of the transverse momentum spectra that gives an approximation to where most of the \pt{} values are located, the variance is studied as a function of the charged-particle multiplicity in the following. The variance gives an estimate to the width of a distribution, i.e. how it is spread along the x-axis. Small variances correspond to slim distributions where most of the values are located close to the mean, large variances correspond therefore to broad distributions where most of the values are spread widely from the mean value.

The upper panels of figure \ref{ue:var} show the variance of the transverse momentum distributions as a function of charged-particle multiplicity for the three regions as well as \textit{all regions} combined with and without the leading particle \pt{} cut applied. The corresponding ratios are shown in the lower panel. The distributions have been scaled to their azimuthal coverage as well, and the unscaled distributions can be found in appendix \ref{app:unscaled}. \\
The distribution measured in the towards region, and without the leading \pt{} cut applied has a similar shape to the MB distribution, as can be discerned from the ratio. It is also considerably enhanced in comparison to the away and the transverse side. It can be concluded that the towards side contains a wider range of measured particle \pt{}, this is also supportive of the assumption that the underlying event is evenly spread in the whole angular region and the jets resulting from the primary collision are superimposed over it. The distributions in the towards region with the leading \pt{} cut applied is also significantly biased by the introduction of the cut. The variance stabilises at $N_\text{} \approx 20$, i.e. $N_\text{ch}/(2\pi\Delta\varphi) \approx 60$, and is roughly constant within the systematic uncertainties towards higher multiplicities. It is significantly enhanced in comparison to the away and the transverse side, the variance is roughly nine times larger than in the transverse region. Correspondingly the distributions in the towards side are three times as broad as in the transverse region. This is again due to the high \pt{} particles resulting from the primary collision. \\
The variance in the away and the transverse region calculated from the event sample with the leading \pt{} cut applied have a similar shape up to $N_\text{ch} \approx 15$, i.e. $N_\text{ch}/(2\pi\Delta\varphi) \approx 45$. This is again evidence that the hadronisation of the primary collision and the MPI share the same mechanisms, but happen at different momentum scales. Above $N_\text{ch} \approx 15$, the variance \pt{} distribution measured in the away region starts increasing with rising multiplicity. It can, however, not be excluded that this is an effect introduced by the unfolding procedure, explained in section \ref{dataset:unfolding}, as is indicated by the large systematical uncertainties.

\section{Summed $p_T$}

In the above sections the transverse momenta of the final state particles as produced in proton-proton collisions is studied in combination with the charged-particle multiplicity. Both of these observables are sensitive to the number and the momentum scale of MPI and the soft processes in the hadronisation phase. 
    
    \begin{figure}
    \centering
	\begin{subfigure}{0.495\textwidth}
		\centering
		\includegraphics[width=0.95\textwidth]{Analysis/img/ue/event_vars/pTsummed.pdf}
	\end{subfigure}
	\hfill
	\begin{subfigure}{0.495\textwidth}
		\centering
		\includegraphics[width=0.95\textwidth]{Analysis/img/ue/event_vars/pTsummedMB.pdf}
	\end{subfigure}
    \caption{}
    \label{ue:sum_pt}
\end{figure}



\begin{figure*}[t!]

    \vspace*{\floatsep}
    \centering
	\begin{subfigure}{0.495\textwidth}
		\centering
		\includegraphics[width=0.95\textwidth]{Analysis/img/ue/correlations/sumComp_all.pdf}
	\end{subfigure}
	\hfill
	\begin{subfigure}{0.495\textwidth}
		\centering
		\includegraphics[width=0.95\textwidth]{Analysis/img/ue/correlations/sumMultComp_all.pdf}
	\end{subfigure}
    \caption{}
    \label{ue:sumPtComp_percent}
\end{figure*}

\clearpage



\end{document}