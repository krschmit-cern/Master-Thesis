\documentclass[main.tex]{subfiles}
\begin{document}

The underlying event constitutes of all particles and interactions that do not derive directly from the hardest parton-parton collision, also called primary collision. This includes additional parton-parton collisions and their resulting interactions as well as initial and final state radiation. While the primary collision can be described using perturbation theory \cite{perturbation_theory}, the underlying event may consist of soft interactions where pQCD is not applicable. The underlying event is therefore sensitive to the phenomenological models used to describe soft processes and can be utilised to determine their accuracy. \\
In chapter \ref{Rivet} the comparison of two distinct MCEG PYTHIA 8.24 respectively ANGANTYR and EPOS LHC has shown deviations in the description of particle collisions from measurement. The analysis of transverse momentum and multiplicity spectra in the underlying event region might offer additional insight in the processes responsible for particle production. 
In the following it is therefore attempted to separate a sample of the underlying event using data from pp collisions at 5.02 TeV center-of-mass-energy. 

\section{Leading Particle}

In practice it is not known a priori which particles result from the primary collisions and which compose the underlying event. As explained in chapter \ref{theory:leading_track} the direction of the primary jet, resulting from the primary collision, is approximated by the track with the highest transverse momentum, the leading track. The event is then divided into three parts with equal azimuthal coverage, see left part of figure \ref{ue:phi_parts}. The primary jet is then confined within the towards and away regions, so that the transverse region only contains underlying-event particles, see right part of figure \ref{ue:phi_parts}. Note that the towards and away regions may still contain underlying event particles. \\
As already explained in chapter \ref{theory:leading_track} is the approximation only applicable above a certain value of the leading track transverse momentum. Above this threshold the (mean) transverse multiplicity is independent of the leading track and reaches a plateau (underlying event plateau). The underlying event plateau for this analysis can be seen in the left hand side of figure \ref{ue:leading_track}. As can be seen from the figure, the leading track transverse momentum cut is set to be at 4 GeV/\textit{c}. The right hand side of figure \ref{ue:leading_track} depicts the corresponding probability distribution of the leading track transverse momentum. The figure shows that roughly 99\% of the previously ?? events will be lost through the cut. \\

\begin{figure*}[t!]
    \centering
	\begin{subfigure}{0.495\textwidth}
		\centering
		\includegraphics[width=0.95\textwidth]{Analysis/img/ue/plaetau_plot.png}
	\end{subfigure}
	\hfill
	\begin{subfigure}{0.495\textwidth}
		\centering
		\includegraphics[width=0.95\textwidth]{Analysis/img/ue/leading_track.png}
	\end{subfigure}
    \caption{Mean multiplicity in transverse region as function of leading track transverse momentum, reaching a plateau (underlying event plateau) at 4 GeV/\textit{c} (left) and probability distribution of leading track transverse momentum (right). The leading transverse momentum cut is indicated respectively.}
    \label{ue:leading_track}
\end{figure*}

\section{Effects of Leading $p_T$ Cut}

The leading transverse momentum cut introduces a bias in the towards and possibly the away region to the data. Figure \ref{ue:delta_phi} shows the probability distribution of the angle between an arbitrary track and the leading track ($\angle = \phi_{lead} - \phi_{track}$) with (left) and without (right) leading transverse momentum cut. ?? Waves in lhs ?? The towards side is always the region containing the highest transverse momentum particle, it is therefore expected that the number of produced particles in this region is enhanced, compared to the other two regions. This is confirmed by the rhs of figure \ref{ue:delta_phi}. Since momentum must be conserved in each event and each interaction respectively it is also expected that the particle production in the away region is also slightly enhanced, compared to the transverse region. With introducing the leading transverse momentum cut the probability of finding a particle in the towards region is enhanced by roughly ?10\%? compared to the distribution without cut. The probability distribution in the away region is also considerably enhanced. 

\begin{figure}
    \centering
	\begin{subfigure}{0.495\textwidth}
		\centering
		\includegraphics[width=0.95\textwidth]{Analysis/img/ue/DeltaPhi.pdf}
	\end{subfigure}
	\hfill
	\begin{subfigure}{0.495\textwidth}
		\centering
		\includegraphics[width=0.95\textwidth]{Analysis/img/ue/DeltaPhiMB.pdf}
	\end{subfigure}
    \caption{Probability distribution of angle between any track and the corresponding leading track for events with applied leading track transverse momentum cut (left) and without (right)}
    \label{ue:delta_phi}
    
    \vspace*{\floatsep}
    
\end{figure}

=> mean und var jeweils in den regionen für die cut variationen 


\begin{enumerate}
    \item[-] 
\end{enumerate}

\end{document}