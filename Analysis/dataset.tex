\documentclass[main.tex]{subfiles}
\begin{document}

This work aims to study the particle production in proton-proton as well as proton-Lead collisions. The first part of this thesis focuses on the comparison of proton-proton as well as proton-Lead data generated with different MCEG to measurement. \\
The simulations are done with the \pythia{8.24}, \pythia{Angantyr} and \epos{} event generators using the Monash 2013 and LHC tune respectively. The proton-proton collisions are generated at center-of-mass energies 2.76 TeV, 5.02 TeV, 7 TeV and 13 TeV, the proton-Lead collisions at center-of-mass energies 5.02 TeV and 8.16 TeV. An overview of the generated statistics can be found in appendix \ref{app:data}. \\
These are compared to proton-proton as well as proton-Lead data at the same center-of-mass energies, measured with the ALICE experiment. Since the statistic uncertainties are negligible, only the systematic uncertainties are shown.

The comparison is done in a kinematic range of $0.15\ \text{GeV}/c \leq 10\ \text{GeV}/c$ and pseudorapidity $-0.8 \leq \eta < 0.8$, the standard track quality constraints as explained in detail in \cite{track_cuts} are used. \\
The second part of this theses focuses on the analyses of charged-particle properties in different phase space regions defined by their azimuthal angle with respect to the highest transverse momentum particle. The analysed proton-proton data at center-of-mass energy 5.02 TeV were taken in 2017 by the ALICE experiment, the corresponding runlist can be found in appendix \ref{app:data}. The same track quality constraints as above are used, and all of the datasets were recorded using the MB trigger. \\
For the charged-particle multiplicity, the transverse momentum, the mean \pt{}, variance \pt{} as well as summed \pt{} the systematic uncertainties are calculated using the standard track variations as used in \cite{track_cuts}. For each variation, the largest contribution is chosen. For the total systematic uncertainty the different contributions are added in quadrature. 
An additional contribution to the total systematic uncertainty arises from the unfolding procedure applied to the data in order to correct for detector effects. It is explained in detail in \cite{mario_thesis}. In this work, the unfolding has been applied separately to the azimuthal regions. Since the Monte Carlo data does not fully represent the measured data, especially in the transverse and away regions, the unfolding procedure is not fully accurate. The Monte Carlo closure test for the azimuthal regions is described in the following section. The resulting uncertainties are added to the total systematic uncertainty. In order to differentiate between unfolded and measured raw multiplicity, the charged-particle multiplicity \mult{} is defined as the unfolded multiplicity and the measured raw multiplicity is denoted $N_{acc}$.

\section{Monte Carlo Closure Test} \label{dataset:unfolding}

As explained in chapter \ref{dec_sim}, the detector acceptance and efficiency influence the result of the measurement. In order to correct for these detector effects, detector simulations such as GEANT are used to estimate the effects of the detector on the measurement. The result of this detector simulation can be subjected to the same correction methods as used in the measured data. These corrected data can then be compared to the original simulation input given to the detector simulation in order to judge the quality of the simulation and the correction methods. Such a test is denoted as Monte Carlo closure test. The systematic uncertainties arising from the imperfect detector description is typically calculated by taking the ratio of the corrected to the generated data. In this work the correction method that is applied to the data is a bayesian unfolding approach. \\
The Monte Carlo closure distributions for \mult in the three azimuthal regions as well as their combination \textit{all regions} are shown in figure \ref{dataset:mult_closure_mb} for the event samples without the leading particle \pt{} cut, and in figure \ref{dataset:mult_closure_cut} for the event sample with the leading \pt{} cut applied. The closed circles indicate the ratio was taken between the fully integrated distributions, the closed stars indicate that the ratio was taken between the distributions limited to one value \mult{}, and then averaged over \mult{} with the contributions being weighted with p(\mult{}). The corresponding weighted standard deviation is shown as a filled area. 
The biggest contributions to the systematic uncertainties result from the small multiplicity values, where the unfolded distribution can not reproduce the generated distributions, as well as from the higher multiplicities where the statistical uncertainties dominate the unfolding procedure. In the range from \mult{} = 5 to \mult{} $\approx$ 30 the unfolding works, contributing to the systematic uncertainties with less than 5\%. 


\begin{figure*}[t!]
    \centering
	\begin{subfigure}{0.495\textwidth}
		\centering
		\includegraphics[width=0.95\textwidth]{Analysis/img/ue/closure_test/MCClosureMult_MB_full.pdf}
	\end{subfigure}
	\hfill
	\begin{subfigure}{0.495\textwidth}
		\centering
		\includegraphics[width=0.95\textwidth]{Analysis/img/ue/closure_test/MCClosureMult_MB_toward.pdf}
	\end{subfigure}
    \vspace*{\floatsep}
    \centering
	\begin{subfigure}{0.495\textwidth}
		\centering
		\includegraphics[width=0.95\textwidth]{Analysis/img/ue/closure_test/MCClosureMult_MB_transverse.pdf}
	\end{subfigure}
	\hfill
	\begin{subfigure}{0.495\textwidth}
		\centering
		\includegraphics[width=0.95\textwidth]{Analysis/img/ue/closure_test/MCClosureMult_MB_away.pdf}
	\end{subfigure}
    \caption{The Monte Carlo closure distributions for \mult{} in the three azimuthal regions as well as their combination \textit{all regions} for events no leading particle \pt{} cut has been applied.}
    \label{dataset:mult_closure_mb}
% \end{figure*}
        \vspace*{\floatsep}
% \begin{figure*}[t!]
    \centering
	\begin{subfigure}{0.495\textwidth}
		\centering
		\includegraphics[width=0.95\textwidth]{Analysis/img/ue/closure_test/MCClosureMult_Cut_full.pdf}
	\end{subfigure}
	\hfill
	\begin{subfigure}{0.495\textwidth}
		\centering
		\includegraphics[width=0.95\textwidth]{Analysis/img/ue/closure_test/MCClosureMult_Cut_toward.pdf}
	\end{subfigure}
    \vspace*{\floatsep}
    \centering
	\begin{subfigure}{0.495\textwidth}
		\centering
		\includegraphics[width=0.95\textwidth]{Analysis/img/ue/closure_test/MCClosureMult_Cut_transverse.pdf}
	\end{subfigure}
	\hfill
	\begin{subfigure}{0.495\textwidth}
		\centering
		\includegraphics[width=0.95\textwidth]{Analysis/img/ue/closure_test/MCClosureMult_Cut_away.pdf}
	\end{subfigure}
    \caption{The Monte Carlo closure distributions for \mult{} in the three azimuthal regions as well as their combination \textit{all regions} for events where the leading particle \pt{} cut has been applied.}
    \label{dataset:mult_closure_cut}
\end{figure*}


\begin{figure*}[t!]
    \centering
	\begin{subfigure}{0.495\textwidth}
		\centering
		\includegraphics[width=0.95\textwidth]{Analysis/img/ue/closure_test/MCClosurePt_MB_full.pdf}
	\end{subfigure}
	\hfill
	\begin{subfigure}{0.495\textwidth}
		\centering
		\includegraphics[width=0.95\textwidth]{Analysis/img/ue/closure_test/MCClosurePt_MB_toward.pdf}
	\end{subfigure}
    \vspace*{\floatsep}
    \centering
	\begin{subfigure}{0.495\textwidth}
		\centering
		\includegraphics[width=0.95\textwidth]{Analysis/img/ue/closure_test/MCClosurePt_MB_transverse.pdf}
	\end{subfigure}
	\hfill
	\begin{subfigure}{0.495\textwidth}
		\centering
		\includegraphics[width=0.95\textwidth]{Analysis/img/ue/closure_test/MCClosurePt_MB_away.pdf}
	\end{subfigure}
    \caption{The Monte Carlo closure distributions for \pt{} in the three azimuthal regions as well as their combination \textit{all regions} for events where the no leading particle \pt{} cut has been applied.}
    \label{dataset:pt_closure_mb}
% \end{figure*}
        \vspace*{\floatsep}
% \begin{figure*}[t!]
    \centering
	\begin{subfigure}{0.495\textwidth}
		\centering
		\includegraphics[width=0.95\textwidth]{Analysis/img/ue/closure_test/MCClosurePt_Cut_full.pdf}
	\end{subfigure}
	\hfill
	\begin{subfigure}{0.495\textwidth}
		\centering
		\includegraphics[width=0.95\textwidth]{Analysis/img/ue/closure_test/MCClosurePt_Cut_toward.pdf}
	\end{subfigure}
    \vspace*{\floatsep}
    \centering
	\begin{subfigure}{0.495\textwidth}
		\centering
		\includegraphics[width=0.95\textwidth]{Analysis/img/ue/closure_test/MCClosurePt_Cut_transverse.pdf}
	\end{subfigure}
	\hfill
	\begin{subfigure}{0.495\textwidth}
		\centering
		\includegraphics[width=0.95\textwidth]{Analysis/img/ue/closure_test/MCClosurePt_Cut_away.pdf}
	\end{subfigure}
    \caption{The Monte Carlo closure distributions for \mult{} in the three azimuthal regions as well as their combination \textit{all regions} for events where the leading particle \pt{} cut has been applied.}
    \label{dataset:pt_closure_cut}
\end{figure*}

\end{document}