\documentclass[main.tex]{subfiles}

\begin{document}

The particles resulting from pp as well as p--Pb collisions are produced through the complex interplay of different mechanisms. Monte Carlo event generators (MCEG) model these mechanisms as explained in chapter \ref{theory:MCEG_simstructure}, and serve to fully simulate particle collisions. In order to test the particle production mechanisms modelled in MCEG, the observables calculated from the kinematic properties of the resulting final-state particles are compared to measurements. \\
Charged-particle multiplicity and transverse momentum distributions are closely linked to particle production and hence are sensitive observables to test MCEG. \\
In the following, the models for particle production incorporated in the \pythia{8.24}, \pythia{Angantyr} and \epos{} event generators will be tested for pp and p--Pb collisisons for the center-of-mass energies 2.76, 5.02, 7 and 13 TeV as well as 5.02 and 8.16 TeV respectively. 

\section{pp Collisions}

The charged-particle multiplicities and transverse-momentum spectra resulting from pp collisions simulated with \pythia{8.24} and \epos{} are compared to measurements of the center-of-mass energies of 2.76, 5.02, 7 and 13 TeV in the following. 

\subsection{Charged-Particle Multiplicity}

\begin{figure*}[t!]
    \centering
	\begin{subfigure}{0.495\textwidth}
		\centering
		\includegraphics[width=0.95\textwidth]{Analysis/img/rivet/pp/CompareRivet_mult_energies_PYTHIA.pdf}
	\end{subfigure}
	\hfill
	\begin{subfigure}{0.495\textwidth}
		\centering
		\includegraphics[width=0.95\textwidth]{Analysis/img/rivet/pp/CompareRivet_mult_energies_EPOS.pdf}
	\end{subfigure}
    \caption{Charged-particle multiplicity distributions arising from \pythia{8.24} (left) and \epos{} (right) simulations each in comparison to measurement.}
    \label{rivet:mult_pp}
%\end{figure*}
    \vspace*{\floatsep}
%\begin{figure*}[th]
    \centering
	\begin{subfigure}{0.495\textwidth}
		\centering
		\includegraphics[width=0.95\textwidth]{Analysis/img/rivet/pp/mult_vs_energies_pp.pdf}
	\end{subfigure}
	\hfill
	\begin{subfigure}{0.495\textwidth}
		\centering
		\includegraphics[width=0.95\textwidth]{Analysis/img/rivet/pp/multmax_vs_energies_pp.pdf}
	\end{subfigure}
    \caption{Mean and Variance (left) as well as position and value of maximum (right) of charged-particle multiplicity from measurement, \pythia{8.24} and \epos{} in dependence of center-of-mass energy}
    \label{rivet:fixpoints_pp}
\end{figure*}

The primary charged-particle multiplicity results directly from the hadronisation phase of particle collisions. It is therefore sensitive to changes in the modeling of the hadronisation and the MPI. The event generators \pythia{8.24} and \epos{} differ in the implemented hadronisation models, as explained in chapter \ref{theory:pythia} and \ref{theory:epos} respectively. While \pythia{8.24} utilizes string hadronisation in the full phase space, \epos{} applies this only in regions of low string densities. Regions of high string densities are hadronised using a statistical hadronisation approach.

Figure \ref{rivet:mult_pp} shows the charged-particle multiplicity distributions simulated with \pythia{8.24} (left) and \epos{} (right) respectively in comparison to the unfolded measurement. Figure \ref{rivet:fixpoints_pp} shows the mean and variance (left) as well as the position and the value of the maximum (right) of these multiplicity distributions as function of the center-of-mass energy. Both event generators are in agreement with the measurement within 30\%. 
They reproduce the mean (variance) of the distributions within 2\% (10\%), especially the increase with rising center-of-mass energy is reproduced quite well by both MCEG. 
The position of the maximum at \mbox{\textit{N}\textsubscript{ch} $=$ 2} in \epos{} data is in agreement with measurement, while it is shifted to \mbox{\textit{N}\textsubscript{ch} $=$ 3} in \pythia{8.24} data.
The value of the maximum, however, is in agreement with the measurement within 10\% for both event generators. The ratios between simulated and measured distributions suggest that both MCEG overestimate the downward gradient from the position of the mean up to multiplicities of about 27. 
The ratios between the \pythia{8.24} data and measurement is almost completely overlapping for all center-of-mass energies. This indicates that the evolution of the spectra with increasing energy is represented quite well. Whereas the ratios between the \epos{} data and measurement do not overlap between multiplicities 10 and 27. This suggests that the energy dependence of the distributions is not represented as well in \epos{} as in \pythia{8.24}. Since the multiplicity at center-of-mass energy of 2.76 TeV arising from \epos{} is in agreement with measurement within 5\% up to multiplicity 35, it can be concluded that charged-particle multiplicity might be modelled well at lower center-of-mass energies, but is not properly propagated to higher center-of-mass energies. Since both \pythia{8.24} and \epos{} are tuned to early LHC data and relatively low center-of-mass energies, the energy dependence arises mostly from the implemented models. The \pythia{8.24} hadronisation model therefore seems to represent the energy dependence better.


\subsection{{Transverse Momentum}}

\begin{figure*}[t]
    \centering
	\begin{subfigure}{0.495\textwidth}
		\centering
		\includegraphics[width=0.95\textwidth]{Analysis/img/rivet/pp/CompareRivet_pT_energies_PYTHIA.pdf}
	\end{subfigure}
	\hfill
	\begin{subfigure}{0.495\textwidth}
		\centering
		\includegraphics[width=0.95\textwidth]{Analysis/img/rivet/pp/CompareRivet_pT_energies_EPOS.pdf}
	\end{subfigure}
    \caption{Transverse momentum distributions of charged particles from \pythia{8.24} (left) and \epos{} (right) in comparison to measurement}
    \label{rivet:pT_pp}
\end{figure*}

The charged-particle multiplicity is closely linked to the transverse momentum of the final-state particles. In the string hadronisation model shorter strings lead to less particles with more momentum, and vice versa. This is especially relevant to the interaction of different MPI and the colour reconnection mechanism. The transverse momentum and it's mean and variance as function of $N_\text{ch}$ are therefore studied in the following.

\begin{table}[t]
    \centering
    \begin{tabular}{r|c|c|c}
          & \multicolumn{3}{c}{average number of particles} \\
        \hline
        cm energy & \pythia{8.24} & \epos{} & measurement \\
        \hline
        2.76 TeV & 7.1 & 7.3 & 6.9 \\
        \hline
        5.02 TeV & 8.2 & 8.3 & 7.5 \\
        \hline
        7 TeV & 8.9 & 8.9 & 7.7 \\
        \hline
        13 TeV & 10.4 & 10.3 & 9.3 \\
    \end{tabular}
    \caption{Average number of particles produced by \pythia{8.24} and \epos{} as well as measured in experiment, calculated from transverse-momentum spectra (see figure \ref{rivet:pT_pp}). Note, that the values for both MCEG are identical to the mean multiplicities shown in figure \ref{rivet:fixpoints_pp}. The deviation between the two calculations in measurement is most likely due to the unfolding process sketched in chapter \ref{dataset:unfolding}}
    \label{rivet:tab:mean_nTracks_pp}
\end{table}

Figure \ref{rivet:pT_pp} shows the transverse-momentum distributions of charged particles from \pythia{8.24} (left) and \epos{} (right) in comparison to measurements at different center-of-mass energies. 
The distributions from both MCEG follow a similar shape. While \pythia{8.24} underestimates the probability of particles carrying transverse momenta below \mbox{\textit{p}\textsubscript{T} $=$ 0.25 GeV/\textit{c}}, \epos{} slightly overestimates the measurement in this \textit{p}\textsubscript{T} interval. Both \pythia{8.24} and \epos{} then overestimate the probability distribution up to \mbox{\textit{p}\textsubscript{T} $=$ 1 GeV/\textit{c}}. For values \mbox{\textit{p}\textsubscript{T} $\geq$ 1 GeV/\textit{c}}, the probability is underestimated for center-of-mass energies 2.76 and 5.02 TeV and overestimated for 7 and 13 TeV. The ratios of the simulated distributions to the measured data show that the maximum of the simulated distributions is shifted to slightly higher transverse momenta in comparison to the measurements. The downward gradient for values of \mbox{\textit{p}\textsubscript{T} $<$ 1 GeV/\textit{c}} is, however, steeper in the MCEG data than suggested by the measurement. This trend continues towards higher transverse momenta in the simulated data at center-of-mass energies of 2.76 and 5.02 TeV, such that above \mbox{\textit{p}\textsubscript{T} $=$ 1 GeV/\textit{c}}, the probability distributions are underestimated. This suggests that the transverse-momentum spectra are estimated too soft by both MCEG.  
Table \ref{rivet:tab:mean_nTracks_pp} shows the average number of particles produced in one event calculated from the transverse-momentum distributions shown in figure \ref{rivet:pT_pp} for \pythia{8.24}, \epos{} as well as measurement. It can be regarded that both \pythia{8.24} and \epos{} events contain on average more particles than measured in experiment. Meaning that rather than the total transverse momentum being distributed incorrectly to the produced partons respectively hadrons, it is distributed to too many particles, resulting in the spectra being too soft.
This could be the consequence of too little colour reconnection between the MPI. A higher probability for colour reconnection to occur between the MPI would lead to smaller strings, resulting in harder spectra with less particles.\\
For data measured at center-of-mass energies of 7 and 13 TeV, the ratios of figure \ref{rivet:pT_pp} show an inflexion point at roughly \mbox{\textit{p}\textsubscript{T} $\approx$ 1 GeV/\textit{c}}. For \mbox{\textit{p}\textsubscript{T} $>$ 1 GeV/\textit{c}} the downward gradient changes slope resulting in the \pythia{8.24} and \epos{} data overestimating the measurement over the full transverse-momentum range. This change in slope of the simulated transverse-momentum distributions suggests that above \mbox{\textit{p}\textsubscript{T} $\approx$ 1 GeV/\textit{c}} the particles resulting from the primary hard collision dominate the distributions.
The energy dependence is reproduced about equally well in both event generators. 

\begin{figure*}[t!] 
    \centering
	\begin{subfigure}{0.495\textwidth}
		\centering
		\includegraphics[width=0.95\textwidth]{Analysis/img/rivet/pp/CompareRivet_mean_energies_PYTHIA.pdf}
	\end{subfigure}
	\hfill
	\begin{subfigure}{0.495\textwidth}
		\centering
		\includegraphics[width=0.95\textwidth]{Analysis/img/rivet/pp/CompareRivet_mean_energies_EPOS.pdf}
	\end{subfigure}
    \caption{Mean transverse momentum as function of multiplicity calculated from \pythia{8.24} (left) and \epos{} (right) in comparison to measurement.}
    \label{rivet:mean_pp}
%\end{figure*}
    \vspace*{\floatsep}
%\begin{figure*}[ht]
    \centering
	\begin{subfigure}{0.495\textwidth}
		\centering
		\includegraphics[width=0.95\textwidth]{Analysis/img/rivet/pp/CompareRivet_variance_energies_PYTHIA.pdf}
	\end{subfigure}
	\hfill
	\begin{subfigure}{0.495\textwidth}
		\centering
		\includegraphics[width=0.95\textwidth]{Analysis/img/rivet/pp/CompareRivet_variance_energies_EPOS.pdf}
	\end{subfigure}
    \caption{Variance of transverse momentum as function of multiplicity calculated from \pythia{8.24} (left) and \epos{} (right) in comparison to measurement.}
    \label{rivet:var_pp}
\end{figure*}

The transverse-momentum and the event multiplicity are closely linked to each other through the hadronisation processes. In order to study these correlations, the mean and variance of the transverse momentum distributions are calculated in the interval $0.15 < p_\text{T}\ (\text{GeV}/c) < 10$ as function of charged-particle multiplicity, as shown in figure \ref{rivet:mean_pp} and \ref{rivet:var_pp} respectively for \pythia{8.24} (left) and \epos{} (right) in comparison to measurements. Both event generators are in agreement with measurement within 5\% for the mean and within 10\% for the variance of the transverse-momentum spectra. 
Both \pythia{8.24} and \epos{} underestimate the mean for multiplicities smaller than ten. \pythia{8.24} starts overestimating the mean transverse momentum above \mbox{\textit{N}\textsubscript{ch} $=$ 10}, while \epos{} approaches the measurement up to a multiplicity  of \textit{N}\textsubscript{ch} $=$ 20 and is then in agreement with measurement within 2\%. This suggests that the composition of string and cluster fragmentation works well for events with multiplicities above 20, while it does not work as well for lower multiplicities. Since \pythia{8.24} is in agreement within 1\% at center-of-mass energy 2.76 TeV in the interval \mbox{10 $\leq$ \textit{N}\textsubscript{ch} $<$ 25}, the mean transverse momentum might be reproduced well for smaller center-of-mass energies but not translated well to higher center-of-mass energies.  \\
In contrast to the multiplicity distributions, the energy dependence of the mean transverse momentum is slightly better described in \epos{}. \\
The variance of the transverse momentum distributions is underestimated in both event generators and the energy dependence is reproduced equally well. 

\vspace{2cm}

\section{p--Pb Collisions}

Since lead nuclei consist of several dozen protons and neutrons, p--Pb collisions are effectively the convolution of a few (separate) pp (pn) sub-collisions. The models implemented in the selected MCEG, \pythia{Angantyr} and \epos{} fundamentally differ in their modelling of the relation between pp and p--Pb collisions, c.f. chapter \ref{theory:pythia} and \ref{theory:epos} respectively. While \pythia{Angantyr} treats the p--Pb collision as a superposition of almost independent pp (pn) sub-collisions, \epos{} utilizes a fluid transition between p--Pb and pp collisions. \\
In the following, the observables already analysed for pp collisions are calculated from events simulated with \pythia{Angantyr} and \epos{} and compared to measurements of p--Pb collisions for the center-of-mass energies of 5.02 and 8.16 TeV. 

    
\subsection{Multiplicity}

Figure \ref{rivet:mult_pPb} shows the charged-particle multiplicity distributions generated with \pythia{Angantyr} (left) and \epos{} (right) in comparison to measurements. Both MCEG reproduce the measured multiplicity distributions within 25\% up to a multiplicity of \mbox{\textit{N}\textsubscript{ch} $=$ 60}. While the data simulated using \angantyr{} is in agreement with the measurement within 30\% up to \mbox{\textit{N}\textsubscript{ch} $=$ 100}, the data simulated using \epos{} greatly underestimates the probability of multiplicities \mbox{\textit{N}\textsubscript{ch} $>$ 60}. \pythia{Angantyr} as well as \epos{} show, in addition to the global maximum at low multiplicities, a second much broader maximum at around \mbox{\textit{N}\textsubscript{ch} $\approx$ 25}, that is not visible in the measured data. 

\begin{figure*}[t!]
    \centering
	\begin{subfigure}{0.495\textwidth}
		\centering
		\includegraphics[width=0.95\textwidth]{Analysis/img/rivet/pPb/CompareRivet_mult_energies_ANGANTYR.pdf}
	\end{subfigure}
	\hfill
	\begin{subfigure}{0.495\textwidth}
		\centering
		\includegraphics[width=0.95\textwidth]{Analysis/img/rivet/pPb/CompareRivet_mult_energies_EPOS_HI.pdf}
	\end{subfigure}
    \caption{Charged-particle multiplicity distributions arising from \pythia{Angantyr} (left) and \epos{} (right) simulations each in comparison to measurement.}
    \label{rivet:mult_pPb}
% \end{figure*}
    \vspace*{\floatsep}
% \begin{figure*}
    \centering
	\begin{subfigure}{0.495\textwidth}
		\centering
		\includegraphics[width=0.95\textwidth]{Analysis/img/rivet/pPb/mult_vs_energies_pPb.pdf}
	\end{subfigure}
	\hfill
	\begin{subfigure}{0.495\textwidth}
		\centering
		\includegraphics[width=0.95\textwidth]{Analysis/img/rivet/pPb/multmax_vs_energies_pPb.pdf}
	\end{subfigure}
    \caption{Mean and Variance (left) as well as position and value of maximum (right) of charged-particle multiplicity from measurement, \pythia{Angantyr} and \epos{} in dependence of center-of-mass energy}
    \label{rivet:fixpoints_pPb}
\end{figure*}

The structure in the \epos{} data is most likely caused by the combination of cluster and string fragmentation that is modelled in the event generator. The total multiplicity distribution is effectively the convolution of two separate distributions, one resulting from the corona with its maximum at low multiplicities, and the other one resulting from the core region with its maximum at around \mbox{\textit{N}\textsubscript{ch} $\approx$ 25}. The core region is expected to start contributing to the multiplicity distribution for multiplicities \mbox{\textit{N}\textsubscript{ch} $\geq$ 12} \cite{EPOS_LHC}. The comparison to the measurement suggests, however, that this threshold should be shifted to slightly lower multiplicities. With increasing string densities inside of the core-region, the produced, and hence the total multiplicity also increases. Since the probability of measuring events with high multiplicities is greatly underestimated, it can be concluded that the string densities might not be modelled high enough at these center-of-mass energies.\\
PYTHIA (8.24) \angantyr{} does not make such a clear separation in phase space between using different hadronisation models. The simulation of each one of the pp (pn) sub-collision, however, is divided into the primary hard collision and the corresponding parton shower as well as the underlying event \ref{theory:MCEG}. The multiplicity distributions can hence be viewed as the convolution of the multiplicity distribution resulting from the simulation of the primary hard collisions of each pp (pn) sub-collisions, and the multiplicity distributions resulting from each underlying event contribution. The ratio between MC data and measurement suggests that the transition between events with a lot of underlying event activity and events with little underlying event activity might be too harsh.\\
Figure \ref{rivet:fixpoints_pPb} shows the mean and variance (left) as well as the position and the value of the maximum (right) of these multiplicity distributions as function of the center-of-mass energy. 
Although the shape of the multiplicity distributions is not represented well in either MCEG, \pythia{Angantyr} describes the mean (variance) of the charged-particle multiplicity within 5\% (7\%). Whereas mean (variance) generated with \epos{} are in agreement with data within 7\% (20\%). The positions of the global maximum is not represented well by both MCEG, while the value of the maximum is in agreement with measurement within 10\%. \\
The energy dependence of the multiplicity distributions resulting from \epos{} and \angantyr{} simulations is represented about equally well in both MCEG. 

\subsection{Transverse Momentum}

The transverse-momentum distributions simulated with \pythia{Angantyr} (left) and \epos{} (right) are shown in comparison to measurements in figure \ref{rivet:pT_pPb}. 

\begin{figure*}[t!]
    \centering
	\begin{subfigure}{0.495\textwidth}
		\centering
		\includegraphics[width=0.95\textwidth]{Analysis/img/rivet/pPb/CompareRivet_pT_energies_ANGANTYR.pdf}
	\end{subfigure}
	\hfill
	\begin{subfigure}{0.495\textwidth}
		\centering
		\includegraphics[width=0.95\textwidth]{Analysis/img/rivet/pPb/CompareRivet_pT_energies_EPOS_HI.pdf}
	\end{subfigure}
    \caption{Transverse momentum distributions of charged particles from \pythia{Angantyr} (left) and \epos{} (right) in comparison to measurement}
    \label{rivet:pT_pPb}
\end{figure*}

Whereas the multiplicity distributions from \pythia{Angantyr} and \epos{} are quite similar in shape, the structure of the corresponding transverse-momentum distributions differ quite a lot. While the \epos{} data is in agreement with the measurement within 15\% up to \mbox{\textit{p}\textsubscript{T} $=$ 4 GeV/\textit{c}}, the \pythia{Angantyr} data describes the measurement within (only) 50\%. 
The transverse-momentum distributions generated with \epos{} are slightly overestimated up to \mbox{\textit{p}\textsubscript{T} $\approx$ 1 GeV/\textit{c}}, for \mbox{\textit{p}\textsubscript{T} $>$ 1 GeV/\textit{c}} the spectra are underestimated, meaning that the spectra are slightly too soft in comparison to the measured data. The data generated with \pythia{Angantyr}, however, greatly overestimates the probabilities for particles carrying \mbox{\textit{p}\textsubscript{T} $<$ 1 GeV/\textit{c}}, and greatly underestimates the probabilities for transverse momenta \mbox{\textit{p}\textsubscript{T} $>$ 1 GeV/\textit{c}}, the spectra are hence much too soft, as can also be seen from the ratio to measurement. This is most likely due to the lack of string interactions in the modelling of p--Pb (A-A) collisions in the \angantyr{} model. Each parton-parton collision can be viewed as a flux tube, i.e. string, as explained in chapter \ref{theory:epos}. These flux tubes could potentially overlap. In this case, the strings could repel each other, giving the flux tubes additional transverse momentum, leading to harder \mbox{\textit{p}\textsubscript{T}} spectra. \\
Since these string interactions are included in the \epos{} model of pp, p-Pb and Pb-Pb (A-A) collisions, the transverse-momentum distributions are much harder in comparison to the \pythia{Angantyr} data.\\
Both \pythia{Angantyr} as well as \epos{} describe the energy dependence of the transverse-momentum distributions quite well, even better than those of the multiplicity distributions.

\begin{figure*}[t!]
    \centering
	\begin{subfigure}{0.495\textwidth}
		\centering
		\includegraphics[width=0.95\textwidth]{Analysis/img/rivet/pPb/CompareRivet_mean_energies_ANGANTYR.pdf}
	\end{subfigure}
	\hfill
	\begin{subfigure}{0.495\textwidth}
		\centering
		\includegraphics[width=0.95\textwidth]{Analysis/img/rivet/pPb/CompareRivet_mean_energies_EPOS_HI.pdf}
	\end{subfigure}
    \caption{Mean transverse momentum as function of multiplicity calculated from \pythia{Angantyr} (left) and \epos{} (right) in comparison to measurement.}
    \label{rivet:mean_pPb}
\end{figure*}
    % \vspace*{\floatsep}
\begin{figure*}[ht]
    \centering
	\begin{subfigure}{0.495\textwidth}
		\centering
		\includegraphics[width=0.95\textwidth]{Analysis/img/rivet/pPb/CompareRivet_variance_energies_ANGANTYR.pdf}
	\end{subfigure}
	\hfill
	\begin{subfigure}{0.495\textwidth}
		\centering
		\includegraphics[width=0.95\textwidth]{Analysis/img/rivet/pPb/CompareRivet_variance_energies_EPOS_HI.pdf}
	\end{subfigure}
    \caption{Variance of transverse momentum as function of multiplicity calculated from \pythia{Angantyr} (left) and \epos{} (right) in comparison to measurement.}
    \label{rivet:var_pPb}
\end{figure*}

As already discussed for the case of pp collisions, the transverse-momentum spectra are strongly influenced by the corresponding multiplicity distributions. In order to attempt an analysis of the transverse momenta independently of the shape of the multiplicity distributions, their mean and variance are calculated as function of charged-particle multiplicity as was the case for pp collisions. The results are shown in figure \ref{rivet:mean_pPb} as well as in figure \ref{rivet:var_pPb}.
As could already be expected from the analysis of the \mbox{\textit{p}\textsubscript{T}} spectra generated with \pythia{Angantyr}, the mean as well as the variance are not described well. The mean and variance distributions calculated from the \epos{} data, however, are in agreement within 8\% and 25\% respectively. The general shape of the distributions is not well described for both (central) moments and center-of-mass energies. For multiplicities of \mbox{\textit{N}\textsubscript{ch} $<$ 20}, the mean (variance) calculated from the generated distributions is underestimated and rises steeply and almost linearly to \mbox{\textit{N}\textsubscript{ch} $\approx$ 24}, while the measured distribution shows a rather smooth curvature over the whole \mbox{\textit{p}\textsubscript{T}} range in comparison. Above \mbox{\textit{N}\textsubscript{ch} $\approx$ 30} the \epos{} data displays only a slight upward gradient, the measurement is underestimated in this region as well. In the interval \mbox{20 $\leq$ \textit{N}\textsubscript{ch} $<$ 30} the measurement is overestimated. The change in the slope of the upward gradient of the \epos{} data might roughly indicate the point where the core region starts contributing to the total multiplicity. 

\end{document}