\documentclass[main.tex]{subfiles}
\begin{document}

In this work the particle production in ultra-relativistic particle collisions is studied by analysing the properties of inclusive charged-particle distributions measured in pp and p--Pb collisions with the ALICE experiment at LHC.
The transverse momentum of primary charged-particles, it's mean and variance as well as the charged-particle multiplicity is analysed in the pseudorapidity range \mbox{$-0.8 \leq \eta < 0.8$} in pp collisions % at 2.76 TeV, 5.02 TeV, 7 TeV, and \mbox{13 TeV} 
as well as for p--Pb collisions. % at center-of-mass energies of 5.02 TeV and 8.16 TeV. \\

In the first part of this thesis the current models of particle production are tested by comparing the results of the Monte Carlo event generators \pythia{8.24}, \pythia{Angantyr} and \epos{} to recent measurements of charged particles in the kinematic range \mbox{$0.15\ \text{GeV}/c \leq p_\text{T} < 10\ \text{GeV}/c$}. 
\textsc{Pythia} and \textsc{Epos} fundamentally differ in their interpretation of collective effects in small systems. 
While \textsc{Pythia} implements the string hadronisation model without considering collective effects in the string movements in pp as well as p--Pb collisions, the \textsc{Epos} model employs a statistical hadronsiation model in regions with high string densities.
The comparison is made for the multiplicity, the transverse-momentum distributions as well as the mean and variance transverse momentum as a function of multiplicity, measured in pp collisions at 2.76 TeV, 5.02 TeV, 7 TeV, and \mbox{13 TeV} as well as p--Pb collisions at center-of-mass energies of 5.02 TeV and 8.16 TeV.
It is found that both event generators reproduce the multiplicity distributions in pp and p--Pb collisions within 25\% and the transverse-momentum distributions within 10\% in comparison to the measurements. \\
The \pythia{8.24} and \pythia{Angantyr} models are generally able to reproduce the charged-particle multiplicity in pp as well as \mbox{p--Pb} collisions better than the transverse-momentum especially with regards to their dependence on center-of-mass energy. The results indicate that the transition between the momentum scales of the primary hard collisions and the MPI in the underlying event might be too harsh, potentially corresponding to deficiencies in the modelling of the matter profiles. Additionally, the probability for colour reconnection between the MPI might be set too low, suggested by the softer transverse-momentum distributions in the simulated than in the measured data. \pythia{Angantyr} fails to reproduce the mean transverse-momentum due to the lack of string interactions in the implemented hadronisation models. \\
\epos{} on the other hand generally reproduces the transverse momentum of the charged-particles better than the multiplicity. The data suggests that the separation between the core and the corona region might be too strict, potentially indicating the need for a smoother transition between them. The results of the employed statistical hadronisation model are general in better agreement with the measurement in the case of p--Pb collisions, i.e. higher string densities. 

In the second part, the charged-particles are analysed in three different regions, defined by their azimuthal angle with respect to the highest momentum particle in the event, the leading particle. The direction of the leading particle serves as an approximation of the axis of the jet produced in the hardest momentum-transfer collision of the event. The region transverse to this axis can be identified with the underlying event under the assumption that it is not contaminated by wide angle radiation from the primary collision. In order to ensure this, a lower threshold is introduced to the transverse momentum of the leading particle. 
In this work the observables as mentioned above as well as the summed-\pt{} distributions are analysed in the three azimuthal regions as well as their combination in pp collisions at a center-of-mass energy of 5.02 TeV in the kinematic range \mbox{$0.15\ \text{GeV}/c \leq p_\text{T} < 50\ \text{GeV}/c$} for event samples with and without this threshold applied. In addition, the fraction Z of a given observable to the event total is defined and employed for the charged-particle multiplicity and summed \pt{}. \\
It is found that in addition to the bias towards smaller impact parameters, the leading-particle \pt{} cut might introduce a bias towards asymmetrical initial momenta of the colliding partons, leading to a strong Lorentz boost in the direction of the towards region.  
The idea of particle collision consisting of one hard collision accompanied by several softer parton-parton sub-collision, evenly spread in the azimuthal range, is confirmed by the measurements. Additionally the data supports the assumption that the hadronisation mechanisms of the hard collision and the underlying event are the same, but happen at different momentum scales.
In the regions between multiplicites $N_\text{acc} \approx 10$ an $N_\text{acc} \approx 30$, where the underlying event starts contributing, 10\% of the produced particles and 25\% of the total transverse momentum can be attributed to the primary collision. While the primary collision contributes mainly to the total transverse-momentum, the MPI contribute mainly to the total multiplicity.\\
Finally it is found that an increase in the number of MPI is accompanied by an increase of the momentum transfer of the hard collision. 


\end{document}