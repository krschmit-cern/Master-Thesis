\documentclass[main.tex]{subfiles}
\begin{document}

In this work the particle production in proton-proton collisions is studied by analysing inclusive charged particle measurements as measured by the ALICE TPC sub-detector. 

In the first part of this thesis, the charged-particle multiplicity distributions, the transverse momentum distributions as well as the mean \pt{} and variance \pt{} distributions as a function of multiplicity are compared to corresponding simulations from the MCEG \pythia{8.25}, \pythia{Angantyr} as well as \epos{}, implementing the current models on particle production. The comparison is done for proton-proton collisions at the center-of-mass energies of 2.76 TeV, 5.02 TeV, 7 TeV, and 13 TeV as well as for proton-Lead collisions at center-of-mass energies of 5.02 TeV, and 8.16 TeV in the kinematic range of $0.15\ \text{GeV}/c \leq p_\text{T} < 10\ \text{GeV}/c$ and pseudorapidity $-0.8 \leq \eta < 0.8$.
It is found that both MCEG describe the distributions about equally well. \pythia{8.24} reproduces the energy dependence of the multiplicity better, while \epos{} is in better agreement for the energy dependence of the transverse momentum spectra and it's derived quantities. In the case of p--Pb collision, the \pythia{Angantyr} event generator fails to describe the measured transverse momentum distributions, this is due to the missing string interactions such as string shoving that. \epos{} implements these and is therefore more successful in describing the data. 

In the second part, each event is divided into three different regions defined by their azimuthal angle with respect to the highest transverse momentum particle, the leading particle. The same observables as described above, measured in proton-proton collisions at a center-of-mass energy of 5.02 TeV,  are analysed in each of the regions respectively in the kinematic of $0.15\ \text{GeV}/c \leq p_\text{T} < 50\ \text{GeV}/c$ and pseudorapidity $-0.8 \leq \eta < 0.8$. Additionally the summed \pt{} distributions, i.e. the total transverse momentum, measured in each of the regions are analysed. The fraction Z of a given observable to the event total is defined as $Z(x; y_\text{total}) = {[\text{mean} (x_\text{region})]_{y_\text{total}}}/{x_\text{total}}$, with $y_\text{total}$ being a condition on the analysed events such as having a specific total multiplicity. The fractions of the multiplicity as well as summed \pt{} are analysed as a function of total multiplicity and summed \pt{}. \\
It is found that the idea of particle collision consisting of one hard collision accompanied by several softer parton-parton sub-collision can be confirmed by the measurements. The measurements support additionally the assumption that the hadronisation mechanisms of the hard collision and MPI are identical, but happen at different momentum scales. Finally it is found that an increase in the number of MPI is accompanied by an increase of the momentum transfer of the hard collision. 


\end{document}