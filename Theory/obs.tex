\documentclass[main.tex]{subfiles}
\begin{document}

The particle production in particle collisions is influenced by various aspects as explained in chapter \ref{theory:pp_collisions}. The biggest uncertainties in the current understanding are tied to the MPI and their interaction, as well as the hadronisation phase of the collision. These are closely related to the number of the produced particles and their momenta. In particular, the correlation between the charged-particle multiplicity and the transverse momentum distributions is sensitive to the hadronisation mechanisms. As a starting point for the analysis described in this thesis, previous measurements of the mean transverse-momentum as function of multiplicity in different collision systems and at different center-of-mass energies with ALICE are, therefore, discussed in the next section. \\
The underlying event on the other hand is sensitive to the MPI and their interactions. The number of MPI influences mainly the charged-particle multiplicity of the underlying event. In combination with colour reconnection effects, however, additional MPI can also cause harder transverse momentum distributions. Both aspects are additionally influenced by the momentum scales of the corresponding elementary processes.
The selection of underlying event samples from measured data is therefore discussed in section \ref{obs:underlying_event}. 

\section{Mean Transverse Momentum} \label{obs:mean_var}

During the parton shower and the hadronisation phase of the collision, part of the invariant mass of the collision system is used for the production of new hadrons, the rest is attributed to the newly produced particles as kinetic energy. 
In the string hadronisation model, as explained in chapter \ref{theory:hadronisation}, the string length is decisive for the distribution of the available energy. Long strings break into many particles with comparably little transverse momentum, while shorter strings produce less particles but more energetic.
The correlation between the transverse momentum and the charged-particle multiplicity is, therefore, sensible to the particle production mechanisms. \\
In order to study this correlation, the transverse-momentum distributions are characterised by their mean, and analysed as a function of charged-particle multiplicity. The mean of a function is a measure of the function's most central value, i.e. it's average. \\
For a discrete distribution $f(x_i)$ the mean can be calculated as follows:

\begin{equation}
    \frac{\sum_i f(x_i)\cdot x_i \cdot w_i}{\sum_i f(x_i) \cdot w_i} \label{eq:mean}
\end{equation}

The factor $w_i$ denotes an additional weight that can be introduced.

\begin{figure}[t!]
    \centering
    \begin{subfigure}{0.495\textwidth}
    \centering
    \includegraphics[width=.9\textwidth]{Theory/img/mean_pp_paper.jpg}
    \end{subfigure}
    %\hfill
    \begin{subfigure}{0.495\textwidth}
    \centering
    \includegraphics[width=.9\textwidth]{Theory/img/mean_pPb_paper.jpg}
    \end{subfigure}
    \caption{Mean transverse momentum as function of charged-particle multiplicity measured in pp collisions at different center-of-mass energies (left) and in different collision systems (right) with ALICE \cite{meanPtPaper}. Systematic uncertainties are shown as boxes, the statistical uncertainties are negligible.}
    \label{obs:mean_paper}
\end{figure}

The left side of figure \ref{obs:mean_paper} shows the mean transverse momentum as a function of charged-particle multiplicity \mult{} in pp collisions at center-of-mass energies \mbox{0.9 TeV}, 2.76 TeV and 7 TeV, measured with the ALICE apparatus during the first run of the LHC.
An increase of the mean transverse momentum with rising multiplicity is observed that is attributed to the effects of colour reconnection between MPI \cite{evidenceCR}. In the case of non-interacting MPI, the mean transverse momentum would reach a plateau at high multiplicities. Colour reconnection, however, causes the string lengths to be shorter than in the non-interacting scenario. The shorter string lengths lead to less particles, carrying more transverse momentum, and in turn to harder transverse-momentum distributions. 
Additionally an increase of the mean transverse momentum with rising center-of-mass energy is observed at fixed values of multiplicity. \\
The right side of figure \ref{obs:mean_paper} shows the mean transverse momentum as a function of multiplicity measured in pp collisions at a center-of-mass energy of 7 TeV in comparison to the mean transverse-momentum distributions measured in p--Pb collisions at 5.02 TeV and Pb--Pb collisions at 2.76 TeV center-of-mass energy per nucleon pair. In p--Pb collisions the mean transverse momentum shows a similar increase with multiplicity to the distributions in pp collisions up to $N_\text{ch} \approx 14$. For higher multiplicities the slope of the distribution decreases, so that the mean transverse momentum is lower than in pp collisions. The slope is, however, steeper than expected under the assumption that p--Pb collisions are superpositions of independent pp collisions \cite{meanPtPaper}. 
In Pb--Pb collisions the increase with rising multiplicity is moderate, with the maximum value being substantially lower than that in pp collisions \cite{meanPtPaper}. This behaviour is usually interpreted in a hydrodynamical picture \cite{hydroPbPb}.  

In addition to the mean of the transverse-momentum distributions, in this work the variance is used to characterise the correlation between transverse momentum and charged-particle multiplicity. The variance is a measure of a functions width, i.e. how closely the values are spread around the mean value. The variance of a discrete distribution $f(x_i)$ is defined as:

\begin{equation}
    \frac{\sum_i f(x_i)\cdot (x_i - \mu)^2 \cdot w_i}{\sum_i f(x_i) \cdot w_i}
\end{equation}

The factor $w_i$ denotes an additional weight, $\mu$ denotes the mean of the function as defined in equation \ref{eq:mean}.

\section{Underlying Event} \label{obs:underlying_event}

In high-energy particle collisions, it is expected to find on average two back-to-back particle jets, accompanied by the underlying event, several soft particles filling the region outside of the jet cores, as explained in chapter \ref{theory:MPI}. The particle jets result from the hardest parton-parton sub-collision in the event, while the underlying event results from MPI and from their interactions in particular. \\
The properties of particle jets have been studied extensively in $e^+e^-$ collider experiments, and their hadronisation can be reproduced well with MCEG.
In pp collisions, however, the composite nature of the protons give rise to additional mechanisms such as MPI, colour-reconnection and string interactions. In order to study the MPI and especially their hadronisation, the properties of the underlying event are probed. 

\begin{figure}[t!]
    \centering
    \includegraphics[width=.5\textwidth]{Theory/img/regions_wheel.jpg}
    \caption{Illustration of the towards, transverse and away regions with respect to the leading particle (red), recreated from \cite{OG_UEMonashPaper}.}
    \label{obs:regions_wheel}
\end{figure}

In principle the underlying event is defined as all particles not resulting directly from the hard collisions, i.e. all particles not belonging to the jets. In practice it is not possible to identify exactly which hadrons result from the hard collision and which result from MPI. Instead operational definitions are employed that associate the underlying event with the phase space outside of the jet cores. 
In this thesis a purely geometrical definition is used, as described in \cite{geomUEdefinition}: \\
The central axis of the (back-to-back) jets is approximated by the direction of the highest $p_\text{T}$ particle, called leading particle. The phase space is then divided into three regions, defined by their azimuthal angle $\Delta\varphi$ with respect to the leading particle as depicted in figure \ref{obs:regions_wheel}. The majority of the jet particles are then contained inside the towards region ($|\Delta\varphi| < 60^\circ$), and the away region ($|\Delta\varphi| > 120^\circ$). The transverse region ($60^\circ \leq |\Delta\varphi| \leq 120^\circ$) hence consists of almost pure underlying event. \\
The jet located in the towards region is often denoted leading jet, as it contains the higher transverse momenta in comparison to the away-side jet. In contrast to $e^+e^-$ collisions, where the momenta of the colliding leptons are fixed at the same value, the momenta of the colliding partons may be very different from each other, leading to a Lorentz-boost in the direction of the leading jet.

\begin{figure}
    \centering
    \includegraphics[width=.7\textwidth, clip=true, trim = 14.8cm 0cm 0cm 0cm]{Theory/img/UEplateau.jpg}
    \caption{Mean inclusive multiplicity in the transverse region as a function of leading particle $p_\text{T}$, here denoted as leading track-jet $p_\text{T}$ \cite{OG_UEMonashPaper}.}
    \label{obs:plateau}
\end{figure}

The basic geometrical selection as described above relies on three assumptions in order to get a clean sample of the underlying event:

\begin{enumerate}
    \item The hardest sub-collision produces at least one reconstructable jet, i.e. is significantly harder than the underlying event.
    \item The direction of this jet can be approximated with the leading particle.
    \item The transverse side is not significantly contaminated with wide angle radiation from the jets.
\end{enumerate}

Especially the first assumption is only valid in events, where the colliding hadrons (almost) fully overlap. In order to ensure that the transverse region can serve as a sample of the underlying event, a lower threshold on the leading particle $p_\text{T}$ is introduced, in the following denoted as leading (particle) \pt{} cut. Figure \ref{obs:plateau} shows the mean inclusive multiplicity $\langle N_\text{inc.} \rangle$ measured in the transverse region as a function of the leading particle $p_\text{T}$ as estimated by various MCEG. In the interval up to leading particle $p_\text{T}$ of 10 GeV/$c$, the mean inclusive multiplicity strongly increases. This is understood to be due to the aforementioned wide angle radiation \cite{OG_UEMonashPaper}. For higher values of the leading particle $p_\text{T}$, the mean multiplicity $\langle N_\text{inc.} \rangle$ is roughly constant. Inside of this plateau, the contributions from the jets to the transverse region are negligible. In events where the leading particle \pt{} lies within the plateau, the transverse region can, therefore, be identified with the underlying event.
Since the probability for hard parton-parton interactions increases with the overlap of the colliding hadrons, the lower leading particle $p_\text{T}$ threshold corresponds to a bias to bigger overlaps, i.e. smaller impact parameters. 
Accordingly the probability for MPI increases with smaller impact parameters, leading to on average higher multiplicities than measured in MB event samples without leading particle \pt{} cut.



\end{document}