\documentclass[main.tex]{subfiles}
\begin{document}

\section{Monte Carlo Event Generators} \label{mceg}

Proton-proton collisions result in a sequence of complex mechanisms reaching from primary interactions to the formation of hadrons, as described above. In order to test the theories and models describing these mechanisms they are implemented in software libraries called Monte Carlo event generators (MCEG). MCEG fully simulate the chained mechanisms involved in particle collisions and provide the derived kinematic properties of the final-state particles as a result. These kinematic properties can be compared to similar observables obtained from measurements in order to conclude whether the implemented theories and models are viable. It is also possible to compare the validity and accuracy of different models in this way by collating the results from different MCEG. \\ 
In general MCEG divide the simulation of particle collisions into five phases as depicted in figure \ref{theory:MCEG}. 

\begin{figure}
    \centering
    \includegraphics[width=.7\linewidth]{Theory/img/MCEG1.png}
    \caption{Caption}
    \label{theory:MCEG}
\end{figure}

\subsection{PYTHIA} \label{gendiff}

\end{document}